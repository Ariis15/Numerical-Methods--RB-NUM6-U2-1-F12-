\documentclass[a4wide,10pt]{article}
\usepackage{a4wide}
\usepackage[applemac,utf8]{inputenc}
\usepackage[danish]{babel}
\usepackage[T1]{fontenc}
%\usepackage{pdfsync}
\usepackage{amsmath,amssymb,amsfonts} 
\usepackage[pdftex]{graphicx}
\usepackage{wrapfig}
\usepackage{color}
\usepackage[small,bf]{caption}

\begin{document}
\title{Numerical Methods: Notater}
\author{Nis Sarup}
\date{\today}
\maketitle

\pagebreak

\section{Lektion 1} % (fold)
\label{sec:lektion_1}

\subsection{Introduktion} % (fold)
\label{sub:introduktion}
Formål:
\begin{itemize}
	\item Lære hvad der findes af vigtige numeriske metoder.
	\item Være i stand til at give et svar inklusiv hvor stor fejlen på det savr er, fejlvurdering
	\begin{itemize}
		\item To fejlkilder
		\begin{itemize}
			\item Afrundingsfejl
			\item Metodefejl
		\end{itemize}
		\item Er resultatet overhovedet rigtigt?
		\begin{itemize}
			\item Fejl 40
			\item Andre fejl
		\end{itemize}
	\end{itemize}
	\item Få en forhøjet matematisk modenhed
	\item C++ kode
\end{itemize}
% subsection introduktion (end)

\subsection{Numerisk løsning af lineære ligningssystemer} % (fold)
\label{sub:numerisk_loesning_af_ligningssystemer}

\subsubsection{Lineære ligningssystemer} % (fold)
\label{ssub:lineaere_ligningssystemer}
\[\begin{array}{*{20}{c}}
{2{x_1} - 4x_2^2 = 5}&{{x_1} - 2{x_2} + {x_3} = 8}&{\cos {x_1} + {x_2} = 3}\\
{3x_1^2 - 5{x_2} = 7}&{{x_2} - 4{x_3} = 7}&{{x_1} - \sin {x_2} = 5}\\
{}&{{x_1} + 6{x_3} = 8}&{}\\
I&{II}&{III}
\end{array}\]
\begin{itemize}
	\item Ligningerne i del I og II ovenover er ikke lineære på grund af potenserne og cos/sin.
\end{itemize}
% subsubsection lineære_ligningssystemer (end)

\subsubsection{Gaussisk Elimenation} % (fold)
\label{ssub:gaussisk_elimenation}
\[\begin{array}{*{20}{c}}
{2{x_1} + 4{x_2} + 8{x_3} = 1}\\
{4{x_1} + 6{x_2} + 4{x_3} = 2}\\
{ - 2{x_1} + 10{x_2} + 5{x_3} = 3}
\end{array} \Rightarrow \left[ {\begin{array}{*{20}{c}}
2&4&8\\
4&6&4\\
{ - 2}&{10}&5
\end{array} \vdots \begin{array}{*{20}{c}}
1\\
2\\
3
\end{array}} \right] \Rightarrow \left[ {\begin{array}{*{20}{c}}
2&4&8\\
0&{ - 2}&{ - 12}\\
0&{14}&{13}
\end{array} \vdots \begin{array}{*{20}{c}}
1\\
0\\
4
\end{array}} \right] \Rightarrow \left[ {\begin{array}{*{20}{c}}
2&4&8\\
0&{ - 2}&{ - 12}\\
0&0&{ - 71}
\end{array} \vdots \begin{array}{*{20}{c}}
1\\
0\\
4
\end{array}} \right]\]
\begin{itemize}
	\item Lav nuller under diagonalen
\end{itemize}
\[\begin{array}{l}
{x_3} =  - {4 \mathord{\left/
 {\vphantom {4 {71}}} \right.
 \kern-\nulldelimiterspace} {71}}\\
{x_2} = \frac{{0 + 12( - {4 \mathord{\left/
 {\vphantom {4 {71}}} \right.
 \kern-\nulldelimiterspace} {71}})}}{{ - 2}} = \frac{{24}}{{71}}\\
{x_1} = {7 \mathord{\left/
 {\vphantom {7 {142}}} \right.
 \kern-\nulldelimiterspace} {142}}
\end{array}\]
\begin{itemize}
	\item Pivoter hvis der kommer for mange nuller.
	\item I numerisk forstand er det et problem hvis der kommer meget små tal ~ $10^{-37}$
	\item Pivot: Find den største under og byt rækken om.
	\item Man kan ikke gøre meget ved det hvis de allesammen er små.
	\item Tip: Brug fornuftige størrelser, ikke nødvendigvis SI-enheder.
\end{itemize}
% subsubsection gaussisk_elimenation (end)

\subsubsection{LU faktorisering} % (fold)
\label{ssub:lu_faktorisering}
\begin{itemize}
	\item $\mathbf{A}x=\mathbf{b}$
	\item $\textbf{A}$ er en $n \cdot n$ matrice.
	\item Samme matrix men forskellige højresider: $\mathbf{A}x=\mathbf{c}$
	\item Lave matricen for sig og højresiden for sig
	\item Gemmer informationerne om hvad der skal gøres med højreside i $\mathbf{L}$
\end{itemize}
% subsubsection lu_faktorisering (end)

% subsection numerisk_løsning_af_ligningssystemer (end)

% section lektion_1 (end)

\end{document}
